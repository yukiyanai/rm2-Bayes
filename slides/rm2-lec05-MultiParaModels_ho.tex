\documentclass[dvipdfmx,handout]{beamer}
\usepackage{myteaching}

\usetheme[nonav,simplefoot,jp,nologo]{Kobe}
\usecolortheme{uriboBlack}

\title{政治学方法論 II}
\subtitle{第5回:多変量の推定}
\author{矢内 勇生}
\institute{法学部・法学研究科}
\date{2015年5月13日}

\begin{document}

 {%% TITTLE PAGE
  \setbeamertemplate{footline}{}
  \frame{\titlepage}
 }

 \begin{frame}{今日の内容}
  \tableofcontents
 \end{frame}

 \section{多変量のベイズ推定}
 
 \subsection{1変量の推定から多変量の推定へ}

 \begin{frame}{1変量の推定}
  \begin{itemize}
   \item 推定の対象:$\theta$
   \item 尤度(サンプリングモデル):$p(y|\theta)$
   \item 事前分布:$p(\theta)$
   \item 事後分布:\pause
         \[
          p(\theta | y) \propto p(y | \theta) p(\theta)
         \]
  \end{itemize}
 \end{frame}

 \begin{frame}{多変量の推定:2母数の場合}
  \begin{itemize}
   \item 推定の対象:$\bm{\theta} = (\theta_1, \theta_2)$
   \item 尤度(サンプリングモデル):$p(y|\bm{\theta})$
   \item 事前分布:$p(\bm{\theta})$
   \item 事後分布:\pause
         \[
          p(\bm{\theta} | y) \propto p(y | \bm{\theta}) p(\bm{\theta})
         \]
         \pause          
         \[
          p(\theta_1, \theta_2 | y) \propto p(y | \theta_1, \theta_2) p(\theta_1, \theta_2)
         \]
   \item \alert{同時分布 (joint distribution)}を考える
  \end{itemize}
 \end{frame}

 \begin{frame}{同時分布:相関がない場合}
   \centering
  \includegraphics[scale=0.5]{../teach-bayes-R/contour1.pdf}
 \end{frame}

 \begin{frame}{同時分布:相関がある場合}
   \centering
  \includegraphics[scale=0.5]{../teach-bayes-R/contour2.pdf}
 \end{frame}

 

 \begin{frame}{興味の対象となる母数と対象とならない母数}
  \begin{itemize}
   \item $\bm{\theta} = (\theta_1, \theta_2)$ のうち、$\theta_1$ のみに
         興味があるとする
   \item 例: $y \sim \mathrm{N}(\mu, \sigma^2)$ で、$\mu$ を知りたい
   \item $\theta_2$:局外(撹乱)母数 (nuisance parameter) と呼ぶ
   \item 推定の目標:$p(\theta_1 | y)$
   \item ベイズルール:
         \[
          p(\theta_1, \theta_2 | y)\propto p(y | \theta_1, \theta_2) p(\theta_1, \theta_2)
         \]
   \item どうする?
  \end{itemize}
 \end{frame}


 \subsection{局外変数を捨象する}
 
 \begin{frame}{積分する}
  \begin{itemize}
   \item $p(\theta_1, \theta_2 | y)$ を得ることはできる
   \item $p(\theta_1 | y)$ が知りたい
        \pause
   \item 積分する (integrate out $\theta_2$)
        \[
          p(\theta_1 | y) = \int p(\theta_1, \theta_2 | y) d\theta_2
        \]
  \end{itemize}
 \end{frame}

 \begin{frame}{分解して考える}
  \begin{itemize}
   \item 分解してから、積分する
    \[
     p(\theta_1 | y) = \int p(\theta_1 | \theta_2, y)p(\theta_2 | y)d\theta_2
    \]
   \item 右辺の中身
   \begin{itemize}
    \item $p(\theta_1 | \theta_2, y)$:条件付き確率
    \item $p(\theta_2 | y)$:$\theta_2$ のウェイト
   \end{itemize}
  \end{itemize}
 \end{frame}

\end{document}